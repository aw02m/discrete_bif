\documentclass[a4paper,dvipdfmx]{jsarticle}
\usepackage{amsmath, amssymb}
\usepackage{txfonts}
\usepackage[dvipdfmx]{graphicx}
\usepackage{bm}

\graphicspath{{./figs/}}

\newcommand{\diff}[3]{
 \ifnum #1<1
 \text{Args[0] should be more than 1} \else{
 \ifnum #1=1 \displaystyle \frac{d#2}{d#3}\else
 \displaystyle \frac{d^{#1} #2}{d#3^{#1}}\fi
 }\fi
}

\newcommand{\pdiff}[3]{
 \ifnum #1<1
 \text{Args[0] should be more than 1} \else{
 \ifnum #1=1 \displaystyle \frac{\partial #2}{\partial #3}\else
 \displaystyle \frac{\partial^{#1} #2}{\partial #3^{#1}}\fi
 }\fi
}

\newcommand{\ppdiff}[3]{
    \displaystyle\frac{\partial^{2} #1}{\partial #2 \partial #3}
}



\begin{document}

\title{力学系の汎用分岐解析手法}

\author{EBIN ES}
% \author{Seiya Amoh${}^\dag$ and Tetsushi Ueta${}^\dag$}

\maketitle

\tableofcontents

\input{intro.tex}
\part{局所的分岐}
\setcounter{section}{0}
\input{discrete.tex}
\input{equilibrium.tex}
\input{autonomous.tex}
\input{nonautonomous.tex}
\newpage
\part{大域的分岐}
\setcounter{section}{0}
\newpage
\part{その他の分岐}
\setcounter{section}{0}
\input{hybrid_autonomous.tex}
\newpage
\part{分岐計算プログラムの設計}
\setcounter{section}{0}
\newpage
\part{付録}
\setcounter{section}{0}
\input{ve.tex}
\newpage

\input{bib.tex}

\end{document}